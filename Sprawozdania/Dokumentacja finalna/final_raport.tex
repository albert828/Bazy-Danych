% !TeX encoding = UTF-8
% !TeX spellcheck = pl_PL

% $Id:$

%Author: Wojciech Domski
%Szablon do ząłożeń projektowych, raportu i dokumentacji z steorwników robotów
%Wersja v.1.0.0
%


%% Konfiguracja:
\newcommand{\kurs}{Bazy danych}
\newcommand{\formakursu}{Projekt}

%odkomentuj właściwy typ projektu, a pozostałe zostaw zakomentowane
%\newcommand{\doctype}{Za\l{}o\.{z}enia projektowe} %etap I
%\newcommand{\doctype}{Raport} %etap II
\newcommand{\doctype}{Dokumentacja} %etap III

%wpisz nazwę projektu
\newcommand{\projectname}{Urządzenie badające mikroklimat w pomieszczeniach}

%wpisz akronim projektu
\newcommand{\acronim}{Mikruś}

%wpisz Imię i nazwisko oraz numer albumu
\newcommand{\osobaA}{Albert \textsc{Lis}, 235534}
%w przypadku projektu jednoosobowego usuń zawartość nowej komendy
%\newcommand{\osobaB}{Bart\l{}omiej \textsc{Cabacki}, 123457}

%wpisz termin w formie, jak poniżej dzień, parzystość, godzina
\newcommand{\termin}{Śr 13:05}

%wpisz imię i nazwisko prowadzącego
\newcommand{\prowadzacy}{dr in\.{z}. Paweł \textsc{Drąg}}

\documentclass[10pt, a4paper]{article}

\include{preambula}

	
\begin{document}

\def\tablename{Tabela}	%zmienienie nazwy tabel z Tablica na Tabela

\begin{titlepage}
	\begin{center}
		\textsc{\LARGE \formakursu}\\[1cm]		
		\textsc{\Large \kurs}\\[0.5cm]		
		\rule{\textwidth}{0.08cm}\\[0.4cm]
		{\huge \bfseries \doctype}\\[1cm]
		{\huge \bfseries \projectname}\\[0.5cm]
		{\huge \bfseries \acronim}\\[0.4cm]
		\rule{\textwidth}{0.08cm}\\[1cm]
		
		\begin{flushright} \large
	%	\emph{Skład grupy:}\\
		\osobaA\\[0.4cm]

		
		\emph{Termin: }\termin\\[0.4cm]

		\emph{Prowadzący:} \\
		\prowadzacy \\
		
		\end{flushright}
		
		\vfill
		
		{\large \today}
	\end{center}	
\end{titlepage}

\newpage
\tableofcontents
\section{Opis działania}
Zestaw oprogramowania wraz z mikrokontrolerem pozwala na monitorowanie mikroklimatu (temperatura, ciśnienie) w wcześniej zadeklarowanych pomieszczeniach (Kuchnia, Łazienka etc.). System automatycznie aktualizuje bazę danych najnowszymi pomiarami. Dla komfortu użytkownika została stworzona aplikacja pozwalająca na wyświetlanie interesujących nas pomiarów.
%Obecne we wszystkich dokumentach

\end{document}







































